\documentclass[a4paper,11pt,twoside,openright]{report}
% \documentclass[15pt]{report}

\usepackage[brazil]{babel}    % d� suporte para os termos na l�ngua portuguesa do Brasil
\usepackage[latin1]{inputenc} % d� suporte para caracteres especiais como acentos e cedilha
\usepackage[T1]{fontenc}      % L� a codifica��o de fonte T1 (font encoding default � 0T1).
\usepackage[dvipdfm]{graphicx} % para inclus�o de figuras (png, jpg, gif, bmp)
\usepackage{graphics}          % figuras gr�ficas
\usepackage{color}             % para letras e caixas coloridas
\usepackage{makeidx}           % �ndice remissivo
\usepackage{a4wide}            % correta formata��o da p�gina em A4
\usepackage{setspace}          % para a dist�ncia entre linhas


\begin{document}
\begin{center}
\titulo{\Large {Lista de Exerc\'{\i}cios 05 \\ M\'{e}todos Estat\'{\i}sticos Aplicados as Ci\^{e}ncias Veterin\'{a}rias}
             \\ Professor Wagner Tassinari 
             \\ \small{E-mail: \textit{wtassinari@gmail.com}}}
\end{center}  

% \pagebreak

% \begin{verbatim}
%  http://groups.google.com/group/estatistica-ciencias-veterinarias?hl=pt-BR
% \end{verbatim

\section*{Exerccio 1}
\hspace{0.5cm}Os dados obtidos num experimento inteiramente ao acaso est\~{a}o apresentados na tabela seguinte. Calcule as m\'{e}dias e
fa\c{c}a um gr\'{a}fico (sugest\~{a}o: BoxPlot) para verificar a dispers\~{a}o dos dados em torno da m\'{e}dia de cada tratamento. 
Interprete os resultados.

\begin{center}
\begin{tabular}{l|l|l|l|l}
\hline
\multicolumn{5}{c}{Tratamento} \\ 
\hline
\multicolumn{1}{c|}{A} & \multicolumn{1}{c|}{B} & \multicolumn{1}{c|}{C} & \multicolumn{1}{c|}{D} & \multicolumn{1}{c}{E} \\ 
\hline
\multicolumn{1}{c|}{12} & \multicolumn{1}{c|}{11} & \multicolumn{1}{c|}{8} & \multicolumn{1}{c|}{15} & \multicolumn{1}{c}{16} \\ 
\multicolumn{1}{c|}{13} & \multicolumn{1}{c|}{8} & \multicolumn{1}{c|}{11} & \multicolumn{1}{c|}{17} & \multicolumn{1}{c}{17} \\ 
\multicolumn{1}{c|}{10} & \multicolumn{1}{c|}{7} & \multicolumn{1}{c|}{13} & \multicolumn{1}{c|}{17} & \multicolumn{1}{c}{19} \\ 
\multicolumn{1}{c|}{13} & \multicolumn{1}{c|}{9} & \multicolumn{1}{c|}{12} & \multicolumn{1}{c|}{17} & \multicolumn{1}{c}{16} \\ 
\multicolumn{1}{c|}{13} & \multicolumn{1}{c|}{9} & \multicolumn{1}{c|}{12} & \multicolumn{1}{c|}{14} & \multicolumn{1}{c}{16} \\ 
\multicolumn{1}{c|}{11} & \multicolumn{1}{c|}{10} & \multicolumn{1}{c|}{10} & \multicolumn{1}{c|}{16} & \multicolumn{1}{c}{18} \\ 
\hline
\end{tabular}
\end{center}

\section*{Exerc\'icio 2}
\hspace{0.5cm}Para verificar se quatro marcas comerciais diferentes que produzem um mesmo produto vendido em embalagens que especificam peso
l\'{\i}quido de 120 gramas cont\'{e}m o peso especificado, um instituto de pesquisa comprou cinco itens de cada marca e os pesou. A tabela abaixo
mostra os pesos em gramas do produto, segundo a marca comercial. Fa\c{c}a uma an\'{a}lise de vari\^{a}ncia e interprete o resultado.

\begin{center}
\begin{tabular}{l|l|l|l}
\hline
\multicolumn{4}{c}{Marca Comercial} \\ 
\hline
\multicolumn{1}{c|}{A} & \multicolumn{1}{c|}{B} & \multicolumn{1}{c|}{C} & \multicolumn{1}{c}{D} \\ 
\hline
\multicolumn{1}{c|}{117} & \multicolumn{1}{c|}{115} & \multicolumn{1}{c|}{118} & \multicolumn{1}{c}{125} \\ 
\multicolumn{1}{c|}{120} & \multicolumn{1}{c|}{110} & \multicolumn{1}{c|}{123} & \multicolumn{1}{c}{121} \\ 
\multicolumn{1}{c|}{114} & \multicolumn{1}{c|}{116} & \multicolumn{1}{c|}{119} & \multicolumn{1}{c}{123} \\ 
\multicolumn{1}{c|}{119} & \multicolumn{1}{c|}{115} & \multicolumn{1}{c|}{122} & \multicolumn{1}{c}{118} \\ 
\multicolumn{1}{c|}{115} & \multicolumn{1}{c|}{114} & \multicolumn{1}{c|}{118} & \multicolumn{1}{c}{118} \\ 
\hline
\end{tabular}
\end{center}

\section*{Exerc\'icio 3}
\hspace{0.5cm}Suponha-se um experimento fict\'{\i}cio de alimenta\c{c}\~{a}o de su\'{\i}nos, em que foram testadas 4 ra\c{c}\~{o}es (A, B, C e D), 
com 3 repeti\c{c}\~{o}es cada uma.  Cada ra\c{c}\~{a}o foi aplicada a conjuntos de quatro animais que se encontrava em baias (constituindo uma unidad
e experimental), sendo que os aumentos de peso obtidos (em kg) encontram-se a seguir.  Fa\c{c}a a an\'{a}lise de vari\^{a}ncia e conclua utilizando-se 5\%
de n\'{\i}vel de signific\^{a}ncia.  Se for o caso compare as m\'{e}dias dos tratamentos pelo teste de Tukey.

\begin{center}
\begin{tabular}{l|l|l|l}
\hline
\multicolumn{1}{c}{} & \multicolumn{3}{c}{Repeti\c{c}\~{o}es} \\ 
\hline
\multicolumn{1}{c|}{Ra\c{c}\~{o}es} & \multicolumn{1}{c|}{1} & \multicolumn{1}{c|}{2} & \multicolumn{1}{c}{3} \\ 
\hline
\multicolumn{1}{c|}{A} & \multicolumn{1}{c|}{35} & \multicolumn{1}{c|}{19} & \multicolumn{1}{c}{30} \\ 
\hline
\multicolumn{1}{c|}{B} & \multicolumn{1}{c|}{40} & \multicolumn{1}{c|}{35} & \multicolumn{1}{c}{45} \\ 
\hline
\multicolumn{1}{c|}{C} & \multicolumn{1}{c|}{39} & \multicolumn{1}{c|}{27} & \multicolumn{1}{c}{21} \\
\hline 
\multicolumn{1}{c|}{D} & \multicolumn{1}{c|}{26} & \multicolumn{1}{c|}{15} & \multicolumn{1}{c}{16} \\ 
\hline
\end{tabular}
\end{center}

\section*{Exerc\'icio 4}
\hspace{0.5cm}Um experimento foi conduzido, no delineamento inteiramente casualizado, com o objetivo de estudar o efeito da aduba\c{c}\~{a}o verde na
 cultura do milho.  Para tanto, foram estudadas 4 leguminosas (A, B, C e D), as quais foram cultivadas com o mesmo n\'{u}mero de plantas por parcela.  
Os resultados em kg de milho por parcela foram:

\begin{center}
\begin{tabular}{l|l|l|l|l|l}
\hline
\multicolumn{1}{c}{} & \multicolumn{4}{c}{Repeti\c{c}\~{o}es} & \multicolumn{1}{c}{} \\ 
\hline
\multicolumn{1}{c|}{Leguminosas} & \multicolumn{1}{c|}{1} & \multicolumn{1}{c|}{2} & \multicolumn{1}{c|}{3} & \multicolumn{1}{c|}{4} & \multicolumn{1}{c}{ $\sum T_{i}$} \\ 
\hline
\multicolumn{1}{c|}{A} & \multicolumn{1}{c|}{10} & \multicolumn{1}{c|}{8} & \multicolumn{1}{c|}{15} & \multicolumn{1}{c|}{7} & \multicolumn{1}{c}{40} \\ 
\hline
\multicolumn{1}{c|}{B} & \multicolumn{1}{c|}{22} & \multicolumn{1}{c|}{28} & \multicolumn{1}{c|}{25} & \multicolumn{1}{c|}{25} & \multicolumn{1}{c}{100} \\ 
\hline
\multicolumn{1}{c|}{C} & \multicolumn{1}{c|}{37} & \multicolumn{1}{c|}{39} & \multicolumn{1}{c|}{42} & \multicolumn{1}{c|}{42} & \multicolumn{1}{c}{160} \\ 
\hline
\multicolumn{1}{c|}{D} & \multicolumn{1}{c|}{45} & \multicolumn{1}{c|}{55} & \multicolumn{1}{c|}{47} & \multicolumn{1}{c|}{53} & \multicolumn{1}{c}{200} \\ 
\hline
\end{tabular}
\end{center}

Fazer a an\'{a}lise de vari\^{a}ncia e concluir com rela\c{c}\~{a}o \`{a} aplica\c{c}\~{a}o do teste \textit{F} e do teste de Tukey. 
Utilizar $\alpha$ = 5\%.

\section*{Exerc\'icio 5}
\hspace{0.5cm}Um experimento foi utilizado para verificar a influ\^{e}ncia da adi\c{c}\~{a}o do horm\^{o}nio de crescimento, tiroxina, para 
o crescimento dos p\^{e}los de chinchilas.   Utilizou-se 3 grupos experimentais (A: controle, ra\c{c}\~{a}o usual, B: ra\c{c}\~{a}o com tiroxina em um
 n\'{\i}vel estipulado e C: ra\c{c}\~{a}o com o dobro desse n\'{\i}vel de tiroxina).  Utilizou-se 30 animais machos e desmamados na mesma semana.  
Foram sorteados 10 animais para cada tratamento.  Ap\'{o}s seis meses de ensaio, avaliou-se o comprimento m\'{e}dio do p\^{e}lo de cada animal 
(unidade experimental) onde se obteve os seguintes resultados (em cm). 

\begin{center}
\begin{tabular}{l|l|l|l|l|l|l|l|l|l|l|l}
\hline
\multicolumn{1}{c}{} & \multicolumn{10}{c}{Repeti\c{c}\~{o}es} & \multicolumn{1}{c}{} \\ 
\hline
\multicolumn{1}{c|}{Tratamentos} & \multicolumn{1}{c|}{1} & \multicolumn{1}{c|}{2} & \multicolumn{1}{c|}{3} & \multicolumn{1}{c|}{4} & \multicolumn{1}{c|}{5} & \multicolumn{1}{c|}{6} & \multicolumn{1}{c|}{7} & \multicolumn{1}{c|}{8} & \multicolumn{1}{c|}{9} & \multicolumn{1}{c|}{10} & \multicolumn{1}{c}{ $\sum T_{i}$} \\ 
\hline
\multicolumn{1}{c|}{A} & \multicolumn{1}{c|}{2,5} & \multicolumn{1}{c|}{2,8} & \multicolumn{1}{c|}{2,3} & \multicolumn{1}{c|}{2,7} & \multicolumn{1}{c|}{2,4} & \multicolumn{1}{c|}{2,8} & \multicolumn{1}{c|}{2,2} & \multicolumn{1}{c|}{2,4} & \multicolumn{1}{c|}{2,6} & \multicolumn{1}{c|}{2,1} & \multicolumn{1}{c}{24,8} \\ 
\multicolumn{1}{c|}{B} & \multicolumn{1}{c|}{2,8} & \multicolumn{1}{c|}{3,5} & \multicolumn{1}{c|}{4,3} & \multicolumn{1}{c|}{2,9} & \multicolumn{1}{c|}{3,3} & \multicolumn{1}{c|}{3,6} & \multicolumn{1}{c|}{3,4} & \multicolumn{1}{c|}{3,7} & \multicolumn{1}{c|}{3,4} & \multicolumn{1}{c|}{3,2} & \multicolumn{1}{c}{34,1} \\ 
\multicolumn{1}{c|}{C} & \multicolumn{1}{c|}{3,5} & \multicolumn{1}{c|}{4,2} & \multicolumn{1}{c|}{3,8} & \multicolumn{1}{c|}{3,9} & \multicolumn{1}{c|}{4,1} & \multicolumn{1}{c|}{4,1} & \multicolumn{1}{c|}{3,2} & \multicolumn{1}{c|}{3,7} & \multicolumn{1}{c|}{4,0} & \multicolumn{1}{c|}{3,8} & \multicolumn{1}{c}{38,3} \\ 
\hline
\end{tabular}
\end{center}

Fa\c{c}a a an\'{a}lise de vari\^{a}ncia e conclua utilizando-se 5\% de n\'{\i}vel de signific\^{a}ncia e interprete os resultados.

% \section*{Exerc\'icio 6}
% \hspace{0.5cm}Num ensaio de variedades de batatinhas, no delineamento inteiramente casualizado, as produ\c{c}\~{o}es foram as seguintes, 
% em kg por parcela de 20 $m^{2}$.
% 
% \begin{center}
% \begin{tabular}{l|l|l|l|l}
% \hline
% \multicolumn{1}{c}{} & \multicolumn{4}{c}{Repeti\c{c}\~{o}es} \\ 
% \hline
% \multicolumn{1}{c|}{Variedades} & \multicolumn{1}{c|}{1} & \multicolumn{1}{c|}{2} & \multicolumn{1}{c|}{3} & \multicolumn{1}{c}{4} \\ 
% \hline
% \multicolumn{1}{c|}{Regente} & \multicolumn{1}{c|}{15,6} & \multicolumn{1}{c|}{18,6} & \multicolumn{1}{c|}{15,2} & \multicolumn{1}{c}{-} \\ 
% \multicolumn{1}{c|}{Rival} & \multicolumn{1}{c|}{21,1} & \multicolumn{1}{c|}{21,7} & \multicolumn{1}{c|}{21,8} & \multicolumn{1}{c}{23,4} \\ 
% \multicolumn{1}{c|}{Patrones} & \multicolumn{1}{c|}{16,4} & \multicolumn{1}{c|}{17,4} & \multicolumn{1}{c|}{18,4} & \multicolumn{1}{c}{19,3} \\ 
% \multicolumn{1}{c|}{Dekama} & \multicolumn{1}{c|}{19,2} & \multicolumn{1}{c|}{21,6} & \multicolumn{1}{c|}{-} & \multicolumn{1}{c}{22,6} \\ 
% \multicolumn{1}{c|}{Fedria} & \multicolumn{1}{c|}{20,4} & \multicolumn{1}{c|}{22,0} & \multicolumn{1}{c|}{23,3} & \multicolumn{1}{c}{21,0} \\ 
% \hline
% \end{tabular}
% \end{center}
% 
% Fa\c{c}a a an\'{a}lise de vari\^{a}ncia e conclua utilizando-se 5\% de n\'{\i}vel de signific\^{a}ncia e interprete os resultados.
% 
% \section*{Exerc\'icio 7}
% \hspace{0.5cm}Obtiveram-se amostras de \'{a}gua de 4 lugares por onde passa um rio, com vistas a se determinar se havia varia\c{c}\~{a}o da 
% quantidade de oxig\^{e}nio dissolvido (uma medida da polui\c{c}\~{a}o dos rios) de um lugar para outro.  Recolheram-se amostras de \'{a}gua de
%  cada um destes lugares, de modo aleat\'{o}rio, que analisadas deram os seguintes valores:
% 
% \begin{center}
% \begin{tabular}{l|l|l|l|l|l|l}
% \hline
% \multicolumn{1}{c}{} & \multicolumn{5}{c}{Conte\'{u}do m\'{e}dio de $O_{2}$ dissolvido} & \multicolumn{1}{c}{} \\ 
% \hline
% \multicolumn{1}{c|}{Localidades} & \multicolumn{1}{c|}{1} & \multicolumn{1}{c|}{2} & \multicolumn{1}{c|}{3} & \multicolumn{1}{c|}{4} & \multicolumn{1}{c|}{5} & \multicolumn{1}{c}{ $\sum T_{i}$} \\ 
% \hline
% \multicolumn{1}{c|}{A} & \multicolumn{1}{c|}{5,9} & \multicolumn{1}{c|}{6,1} & \multicolumn{1}{c|}{6,3} & \multicolumn{1}{c|}{6,1} & \multicolumn{1}{c|}{6,0} & \multicolumn{1}{c}{30,4} \\ 
% \hline
% \multicolumn{1}{c|}{B} & \multicolumn{1}{c|}{6,3} & \multicolumn{1}{c|}{6,6} & \multicolumn{1}{c|}{6,4} & \multicolumn{1}{c|}{6,4} & \multicolumn{1}{c|}{6,4} & \multicolumn{1}{c}{32,1} \\ 
% \hline
% \multicolumn{1}{c|}{C} & \multicolumn{1}{c|}{4,8} & \multicolumn{1}{c|}{4,3} & \multicolumn{1}{c|}{5,0} & \multicolumn{1}{c|}{4,7} & \multicolumn{1}{c|}{5,1} & \multicolumn{1}{c}{23,9} \\ 
% \hline
% \multicolumn{1}{c|}{D} & \multicolumn{1}{c|}{6,0} & \multicolumn{1}{c|}{6,2} & \multicolumn{1}{c|}{6,1} & \multicolumn{1}{c|}{5,8} & \multicolumn{1}{c|}{-} & \multicolumn{1}{c}{24,1} \\ 
% \hline
% \end{tabular}
% \end{center}
% 
% Fa\c{c}a a an\'{a}lise de vari\^{a}ncia e o teste de Tukey, utilizando-se um n\'{\i}vel de 5\% de signific\^{a}ncia e dizer qual(is) a(s) localidade(s) que apresentou(ram)
%  maior \'{\i}ndice de polui\c{c}\~{a}o.

\section*{Quest\~{a}o 6}
\hspace{0.5cm}A tabela abaixo mostra os dados relativos ao peso (kg) e ao comprimento (cm) de 12 cavalos com 3 meses de vida de uma determinada esp\'{e}cie:

\begin{center}
% use packages: array
\newcommand{\mc}[3]{\multicolumn{#1}{#2}{#3}}
%
\begin{tabular}{|l|l|}\hline
\mc{1}{|c|}{Peso} & \mc{1}{|c|}{Comprimento} \\ \hline
\mc{1}{|c|}{70} & \mc{1}{|c|}{155} \\ \hline
\mc{1}{|c|}{63} & \mc{1}{|c|}{150} \\ \hline
\mc{1}{|c|}{72} & \mc{1}{|c|}{180} \\ \hline
\mc{1}{|c|}{60} & \mc{1}{|c|}{135} \\ \hline
\mc{1}{|c|}{66} & \mc{1}{|c|}{156} \\ \hline
\mc{1}{|c|}{70} & \mc{1}{|c|}{168} \\ \hline
\mc{1}{|c|}{74} & \mc{1}{|c|}{178} \\ \hline
\mc{1}{|c|}{65} & \mc{1}{|c|}{160} \\ \hline
\mc{1}{|c|}{62} & \mc{1}{|c|}{132} \\ \hline
\mc{1}{|c|}{67} & \mc{1}{|c|}{145} \\ \hline
\mc{1}{|c|}{65} & \mc{1}{|c|}{139} \\ \hline
\mc{1}{|c|}{68} & \mc{1}{|c|}{152} \\ \hline
\end{tabular}
\end{center}



Pede-se:

\begin{enumerate}
	\item Construa o diagrama de dispers\~{a}o.
	\item Escreva o modelo de regress\~{a}o.
	\item Verifique se o modelo \'{e} adequado. 
	\item Verifique o coeficiente de determina\c{c}\~{a}o ($R^2$).
	\item Em sua opini\~{a}o o modelo empregado explica bem a rela\c{c}\~{a}o entre as vari\'{a}veis ? Por qu\^{e} ?.
\end{enumerate}

\section*{Quest\~{a}o 7}
\hspace{0.5cm}A administra\c{c}\~{a}o de um banco desejava estabelecer um crit\'{e}rio objetivo para avaliar a efici\^{e}ncia de seus gerentes. 
Para isso levantou, pra cada um dos subdistritos onde possuia ag\^{e}ncia, dados a respeito do dep\'{o}sito m\'{e}dio mensal por ag\^{e}ncia e o 
n\'{u}mero de estabelecimentos comerciais existentes nesses subdistritos. Os dados s\~{a}o os seguintes:

\begin{center}
% use packages: array
\begin{tabular}{|c|c|c|}\hline
Subdistritos & X & Y \\ \hline
Nossa Senhora & 16 & 14 \\ \hline
Casa Verde & 30 & 16 \\ \hline
Vila Formosa & 35 & 19 \\ \hline
Santana & 70 & 30 \\ \hline
Barra Funda & 90 & 31 \\ \hline
J. Paulista & 120 & 33 \\ \hline
Santo Amaro & 160 & 35 \\ \hline
Lapa & 237 & 43 \\ \hline
Pinheiros & 378 & 50 \\ \hline
\end{tabular}
\end{center}

Onde, \\
X = N\'{u}mero de Estabelecimentos Comerciais\\
Y = Dep\'{o}sito m\'{e}dio por ag\^{e}ncia (R\$ 10 000)\\
Pede-se:


\begin{enumerate}
	\item Ajustar uma reta aos dados.
	\item Testar a exist\^{e}ncia de regress\~{a}o linear.
	\item Estimar o coeficiente de explica\c{c}\~{a}o ou determina\c{c}\~{a}o ($R^2$).
	\item Interprete os resultados.
\end{enumerate}

\section*{Exerc\'{\i}cio 8}
\hspace{0.5cm}Os dados abaixo s\~{a}o referentes ao consumo de cigarros per capito em 1930 e as mortes por 1.000.000 de habitantes em 1950 causadas
 por c\^{a}ncer no pulm\~{a}o, em 11 pa\'{\i}ses. Pede-se:

\begin{enumerate}
	\item Desenhar o diagrama de dispers\~{a}o;
	\item Calcular o coeficiente de correla\c{c}\~{a}o;
         \item Testar a exist\^{e}ncia de regress\~{a}o linear.
	\item Estimar o coeficiente de explica\c{c}\~{a}o ou determina\c{c}\~{a}o ($R^2$). 
	\item Interprete seus achados.
\end{enumerate}

\begin{center}
\begin{tabular}{l|l|l}
\hline
\multicolumn{1}{c|}{Pa\'{\i}s} & \multicolumn{1}{c|}{Consumo de cigarros} & \multicolumn{1}{c}{Mortes} \\ 
\hline
\multicolumn{1}{c|}{Isl\^{a}ndia} & \multicolumn{1}{c|}{240} & \multicolumn{1}{c}{63} \\ 
\multicolumn{1}{c|}{Noruega} & \multicolumn{1}{c|}{255} & \multicolumn{1}{c}{100} \\ 
\multicolumn{1}{c|}{Su\'{e}cia} & \multicolumn{1}{c|}{340} & \multicolumn{1}{c}{140} \\ 
\multicolumn{1}{c|}{Dinamarca} & \multicolumn{1}{c|}{375} & \multicolumn{1}{c}{175} \\ 
\multicolumn{1}{c|}{Canad\'{a}} & \multicolumn{1}{c|}{510} & \multicolumn{1}{c}{160} \\ 
\multicolumn{1}{c|}{Austria} & \multicolumn{1}{c|}{490} & \multicolumn{1}{c}{180} \\ 
\multicolumn{1}{c|}{Holanda} & \multicolumn{1}{c|}{490} & \multicolumn{1}{c}{250} \\ 
\multicolumn{1}{c|}{Su\'{\i}\c{c}a} & \multicolumn{1}{c|}{180} & \multicolumn{1}{c}{180} \\ 
\multicolumn{1}{c|}{Fil\^{a}ndia} & \multicolumn{1}{c|}{1125} & \multicolumn{1}{c}{360} \\ 
\multicolumn{1}{c|}{Inglaterra} & \multicolumn{1}{c|}{1150} & \multicolumn{1}{c}{470} \\ 
\multicolumn{1}{c|}{EUA} & \multicolumn{1}{c|}{1275} & \multicolumn{1}{c}{200} \\ 
\hline
\end{tabular}
\end{center}


\end{document}